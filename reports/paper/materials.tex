%!TEX root = paper.tex
\section{Plant Seedlings Classification}
\subsection{Data}

\indent{\indent The dataset is a part of the database have been recorded at Aarhus University Flakkebjerg Research station in a collaboration between University of Southern Denmark and Aarhus University. Images are avaliable to researches at \url{https://vision.eng.au.dk/plant-seedlings-dataset/}. The specific of the dataset is that recorded plants are in different growth stages since detecting weed in it's early stage is the thing makes the task problematic. }

\indent{The dataset contains 960 unique plant images of 12 species. The sizes of plant classes are not balanced among themselves - from 221 to 654 labeled samples of each class. Original images are cropped, but have different resolutions from 50x50px to 2000x2000px. Also, images have a different background - some of them on the ground, other on the marked paper.}
\begin{figure}[h]
    \centering
    \includegraphics[height=7.5cm, width=10cm]{first_view_grid_1}
    \caption{Data overview}
    \label{fig:1}
\end{figure}

\subsection{Data preprocessing}
\indent{\indent Firstly, we reduce resolution of all images to the same resolution 200x200px using bilinear interpolation. The main idea of bilinear interpolation that the new image pixel is the weighted sum of neighboring pixels of the original image. It helps to decrease computational complexity and build features.}

\indent{}

\subsection{Feature selection?}

\indent{\indent Features of the images define their content. We recognise the information images provide us with taking into account a great number of features. Then we answer what do we see exactly. The same process can be projected on image classification task: image features let the classifier propose the output decision. Another advantage of the approach is that it reduces feature space for a machine learning algorithm. We often need only a part of the information image is carrying, hence we don't need to process and interpret all the pixels, what can lead to extra computational expences.}

\indent{ Selecting features is a complicated and convoluted research area itself, the assertion if supported by the variety of feature types and the need of presenting essential properties on the equal basis with the previous assertion.}

\indent{ As dicucussed before, we need to define the set of features describing the dataset in the best way. Supposed features must meet the following criterion:}

\begin{itemize}
    \item The feature space should be low-dimensional
    \item The features should not be correlated or be correlated as less as possible
    \item Selected features should represent the content of an image as fully as possible
\end{itemize}

\indent{ We are going to group selected features and define them.}

\subsection{Color features}

\begin{figure}[h]
    \centering
    \includegraphics[height=5.5cm, width=10cm]{to_rgb_sample_1}
    \caption{Data overview}
    \label{fig:2}
\end{figure}

\subsection{Shape features}

\indent{\indent \textbf{Perimeter}}

\textbf{Total area}

\textbf{Maximal contour area}

\textbf{Number of bounding contours}

\textbf{Minimum bounding regtangle}

\textbf{Circularity}


\subsection{Classification}
