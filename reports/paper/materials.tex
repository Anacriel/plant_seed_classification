%!TEX root = paper.tex
\section{Plant Seedlings Classification}
%!TEX root = paper.tex
\subsection{Data}

\indent{\indent The considered dataset is a part of the database that has been recorded at the Aarhus University Flakkebjerg Research station in the collaboration between the University of Southern Denmark and Aarhus University. Images are available to researchers at \url{https://vision.eng.au.dk/plant-seedlings-dataset/}. The specifics of this dataset is that recorded plants are in different growth stages since detecting a weed in its early stage is the thing which makes the task problematic. }

\indent{The dataset contains 960 unique plant images of 12 species. The sizes of plant classes are not balanced among themselves---they range from 221 to 654 labeled samples for each class. Original images are cropped by plant boundaries, but their resolutions vary from 50x50px to 2000x2000px. Also, images have  different backgrounds---some of them are on the ground, other are on the marked paper.}

\begin{figure}[h]
    \centering
    \includegraphics[height=8.5cm, width=10.5cm]{first_view_grid_smaller_1}
    \caption{Data overview}
    \label{fig:1}
\end{figure}

\subsection{Data preprocessing}
\indent{\indent \textbf{Resolution reducing}}

\indent{\indent \textbf{Resolution reducing}}\quad {We reduce the resolution of all images to the same resolution 200x200px using the bilinear interpolation. The main idea of the bilinear interpolation is that a new image pixel is defined as the weighted sum of neighboring pixels of the original image. It helps to decrease computational complexity and build normalized features \cite{bilinear1963interp}.}

\textbf{Segmentation} \quad {The objects of our study are plants, and they are painted green. Therefore, we can create a mask that filters the range of green channel and ignores the other pixels. For these purposes, the HSV (Hue Saturation Value) color model is a suitable representation \cite{hsv1978color}. In the BGR format, the value of each component depends on the amount of light hitting the object. HSV allows us to distinguish between the image color and brightness. We set the lower and upper bounds of the green color using HSV representation. Then we merely mark the pixels in the green range and get a color mask (Fig. \ref{fig_f_ seg_step_2}). Now we apply the operation of logical multiplication to the original image, assign the value of the background pixels to a black color value, and get a segmented plant.}

\threeimage{seg_step_1}{seg_step_2}{seg_step_3}{Source image}{Mask}{Segmented image}

\textbf{Denoising}\quad {Segmentation does not always work well (see Fig. \ref{fig_f_ seg_step_3}). Small areas of the background may fall into the range of green values which distorts the binary mask as in Fig. \ref{fig:seg_no_morph_1}}.

\begin{figure}[h]
	\centering
    \subfigure[Before morphological closure]{
		{\includegraphics[width=5cm, height=5cm]{seg_no_morph_1}
		\label{fig:seg_no_morph_1}}
	}
    \qquad
    \subfigure[After morphological closure]{
		{\includegraphics[width=5cm, height=5cm]{seg_morph_1}
		\label{seg_morph_1}}
	}
    \caption{Segmentation improvement example}%
    \label{fig:seg_improve}
\end{figure}

\indent{Such drawbacks can be eliminated by the morphological operations of the nonlinear transformation associated with the shape and structure of an image. The morphology is used to study the interaction of an image with a specific structural element,  \emph{the kernel}. The kernel iterates over the entire image and compares with the neighborhood of the pixels after which we apply morphological operations \cite{friel2000imanalysis}.}

\indent{To improve segmentation, we use the operation of the morphological closure---a combination of the dilatation and erosion operations \cite{morph2000java}.}

\indent{Then we apply the morphological closure operation to the image (Fig. \ref{fig:seg_no_closure_good_1}) by selecting an elliptical core of 6x6px size and delete the remaining objects within an area of less than 160px. The plant on the image (Fig. \ref{fig:seg_with_closure_bad_1}) has no cavities, and the background is cleared of non-plant elements. But the morphological closure does not always improve the segmentation result. Estimate the result of processing an image of the class Loose silky-bent in Fig. \ref{fig:seg_degradation}: the cavities corresponding to the background were restored that does not correspond to the desired result. We restrict ourselves to the removal of objects whose contours limit a small area.}

\begin{figure}[h]
	\centering
    \subfigure[Before morphological closure]{
		{\includegraphics[width=5cm, height=5cm]{seg_no_closure_good_1}
		\label{fig:seg_no_closure_good_1}}
	}
    \qquad
    \subfigure[After morphological closure]{
		{\includegraphics[width=5cm, height=5cm]{seg_with_closure_bad_1}
		\label{fig:seg_with_closure_bad_1}}
	}
    \caption{Segmentation degradation example}%
    \label{fig:seg_degradation}
\end{figure}


\subsection{Feature selection?}

\indent{\indent Features of the images define their content. We recognise the information images provide us with taking into account a great number of features. Then we answer what do we see exactly. The same process can be projected on image classification task: image features let the classifier propose the output decision. Another advantage of the approach is that it reduces feature space for a machine learning algorithm. We often need only a part of the information image is carrying, hence we don't need to process and interpret all the pixels, what can lead to extra computational expences.}

\indent{ Selecting features is a complicated and convoluted research area itself, the assertion if supported by the variety of feature types and the need of presenting essential properties on the equal basis with the previous assertion.}

\indent{ As dicucussed before, we need to define the set of features describing the dataset in the best way. Supposed features must meet the following criterion:}

\begin{itemize}
    \item The feature space should be low-dimensional
    \item The features should not be correlated or be correlated as less as possible
    \item Selected features should represent the content of an image as fully as possible
\end{itemize}

\indent{ We are going to group selected features and define them.}

\subsection{Color features}

\indent{\indent Overviewing the dataset we can notice that all the plant species are mostly green. Additionally, images were recorded under specific conditions. We will use RGB color model, which stands for red, green and blue colors, and calculate features described below.}

\begin{figure}[h]
    \centering
    \includegraphics[height=5.5cm, width=10cm]{to_rgb_sample_1}
    \caption{RGB transformation}
    \label{fig:2}
\end{figure}

\vspace{3cm}
\indent{Let $\{x^{(k)}\}_{i=1}^N$, where $k = 1, 2, 3 $ –– an index of a channel in RGB color space respectively, $N$ –– total number of the image pixels, $x^{(k)}_i$ –– $i$-th pixel of the $k$-th channel. We will compute sample mean and standard deviation for each channel:}

\begin{equation}
	\label{eq:1}
	\overline{x^{(k)}} = \frac{1}{N}\sum_{i=1}^{N}x^{(k)}_i 
\end{equation}

\begin{equation}
	\label{eq:2}
	 s^{(k)} = \sqrt{\frac{1}{N}\sum_{i=1}^{N}(x^{(k)}_i - \overline{x^{(k)}})^2} 
\end{equation}

\subsection{Shape features}

\indent{\indent \textbf{Number of bounding contours}}

\indent{Segmentation output may result in image divided in separate parts of the plant. We decided to use it and compute the number of bounding contours. We will use the boundary tracing algorithm for the boundary extraction. The designated algorithm \cite{cv1985contours} was implemented in OpenCV \cite{opencv2000python} library for the Python programming language. The studies did not take into account contours bounding areas below a certain threshold, which was empirically chosen.}

\indent{ Let $K$ be the number of detected bounding contours above the threshold in the further contour-related characteristics.}

\vspace{1cm}

\textbf{Total perimeter}

\indent{In this feature we will count the sum of perimeters of all the areas bounded by contours:}

\begin{equation}
	\label{eq:3}
	 P = \sum_{i=1}^{K}p_i,
\end{equation}
where $p_i$ –– $i$-th perimeter

\vspace{1cm}

\textbf{Total area}

\indent{It includes all the areas bounded by contours:}

\begin{equation}
	\label{eq:4}
	 S = \sum_{i=1}^{K}s_i,
\end{equation}
where $s_i$ –– $i$-th area

\vspace{1cm}

\textbf{Maximal contour area}

\indent{We will be analysing the contours bounded maximal areas:}

\begin{equation}
	\label{eq:5}
	 S_m = \max{s_i}, \; i = 1, \dots, K,
\end{equation}
where $s_i$ –– $i$-th area

\vspace{1cm}

\textbf{Rectangularity}

\indent{One of the methods to estimate rectangularity is to plot minimum bounding rectangle...}

\vspace{1cm}

\textbf{Circularity}

\indent{Another title of this shape factor is the isoperimetric quotient and it shows how much area pe perimeter is bounding:}

\begin{equation}
	f_{circ} = \frac{4 \pi A}{P^2},
	\label{eq:6}
\end{equation}
where $P$ –– total perimeter, $A$ –– total area of all detected elements of a plant

\indent{ Consider the correlation matrix of the described features:}

\begin{figure}[h]
	\centering
	\includegraphics[width=12.cm, height=11.5cm]{corr_heatmap_1}
	\caption{Feature correlation matrix}
	\label{fig_corr_matrix}
\end{figure}

\indent{ Based on the data in the figure \ref{fig_corr_matrix}, we conclude that the most linearly dependent features are the total area and the largest area, but this is not true for all classes due to the predominance of plants, limited by the onlyone contour, so the consideration of the maximum area feature is not excluded.}

\subsection{Classification}
\indent{\indent The main method for this task is Support Vector Machine (SVM) –– a binary classification algorithm based on building a separating hyperplane. The algorithm is implemented in Scikit-learn (\cite{scikit2011python}) library for the Python programming language.}

\indent{We will use the Radial Basis Function (RBF) as the kernel function for SVM. The choice was made based on the following advantages:}

\begin{itemize}
	\item RBF usage allows to build a hyperplane when the data is not linearly separable
	\item Only one parameter that affects the resulting solution needs to be tuned
	\item Values of the RBF function are in $(0, 1]$, which does not include zero or infinity
\end{itemize}

\indent{ The SVM algorithm is sensitive to non-normalized data, especially when using the RBF kernel, which is the Euclidian distance itself. In the case when the feature values are in different intervals, a slight difference in one of them can lead to going out of range in second feature values. The solution is to map all the values into one segment, in this task we will choose  $[0, 1]$.}