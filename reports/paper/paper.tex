% This is samplepaper.tex, a sample chapter demonstrating the
% LLNCS macro package for Springer Computer Science proceedings;
% Version 2.20 of 2017/10/04
%
\documentclass[runningheads]{llncs}
%
\usepackage{graphicx}
% Used for displaying a sample figure. If possible, figure files should
% be included in EPS format.
%
% If you use the hyperref package, please uncomment the following line
% to display URLs in blue roman font according to Springer's eBook style:
% \renewcommand\UrlFont{\color{blue}\rmfamily}

\begin{document}
%
%\titlerunning{Abbreviated paper title}
% If the paper title is too long for the running head, you can set
% an abbreviated paper title here
%
\author{Dmitri Jakovlev \and Julia ... \and Georgy Shevlyakov}
% \author{First Author\inst{1}\orcidID{0000-1111-2222-3333} \and
% Second Author\inst{2,3}\orcidID{1111-2222-3333-4444} \and
% Third Author\inst{3}\orcidID{2222--3333-4444-5555}}
%
\authorrunning{F. Author et al.}
% First names are abbreviated in the running head.
% If there are more than two authors, 'et al.' is used.
%
\institute{Peter the Great Saint-Petersburg Polytechnic University, Department of Applied Mathematics, Russia}

% \institute{Princeton University, Princeton NJ 08544, USA \and
% Springer Heidelberg, Tiergartenstr. 17, 69121 Heidelberg, Germany
% \email{lncs@springer.com}\\
% \url{http://www.springer.com/gp/computer-science/lncs} \and
% ABC Institute, Rupert-Karls-University Heidelberg, Heidelberg, Germany\\
% \email{\{abc,lncs\}@uni-heidelberg.de}}
%
\maketitle              % typeset the header of the contribution
%
\begin{abstract}
The abstract should briefly summarize the contents of the paper in
150--250 words.

\keywords{First keyword  \and Second keyword \and Another keyword.}
\end{abstract}
%
%
%
\section{Introduction}


\section{Materials and methods}

\section{Results}

\section{Discussion}

\section{Conclusion}

\section{Acknowledgments?}
% ---- Bibliography ----
%
% BibTeX users should specify bibliography style 'splncs04'.
% References will then be sorted and formatted in the correct style.
%
% \bibliographystyle{splncs04}
% \bibliography{mybibliography}
%
\begin{thebibliography}{}
	\bibitem{bib_1} Giselsson, T., Jørgensen, R., Jensen, P., Dyrmann, M., and Midtiby, H. (2017). A Public Image Database for Benchmark of Plant Seedling Classification Algorithms.
	\bibitem{bib_2} Bradski, G. (2000). The OpenCV Library. Dr. Dobb's Journal of Software Tools.
	\bibitem{bib_3} Oliphant, T. E. (2006). A guide to NumPy (Vol. 1). Trelgol Publishing USA.
	\bibitem{bib_4} Panja, D., Poppe, R. (2018). INFOIBV. Image Processing course, Universiteit Utrecht.
	\bibitem{bib_5} Wojnar, L., Kurzydłowski, K. J. et al. (2000). Practical Guide to Image Analysis, ASM International.
	\bibitem{bib_6} Suzuki, S. and Abe, K. (1985). Topological Structural Analysis of Digitized Binary Images by Border Following.
	\bibitem{bib_7} Pedregosa et al. (2011). Scikit-learn: Machine Learning in Python
	\bibitem{bib_8} Chih-Wei Hsu, Chih-Chung Chang, Chih-Jen Lin (2003). A Practical Guide to Support Vector Classification.
\end{thebibliography}
\end{document}
