%!TEX root = paper.tex
\section{Introduction}

\indent{\indent The demand for agricultural products is increasing day by day, as the population of the Earth is growing. Even though people work on plant classification algorithms, approaches are still not as efficient and robust as desired. A significant part of this work has still been done by people. The question arises of the efficiency with which human resources are used. We will use exhaustible natural resources wisely and increase harvests if we automatize quality assurance, which objectives are to detect and distinguish weeds among the variety of crop seedlings.}

\indent{ All this naturally leads to the idea of automation of the classification process with the help of machine learning algorithms. From recent experience, neural networks are well suited for image processing, but we have to pay for it by computational costs. On the other hand, we could use less costly algorithms, but they require finer tuning to achieve comparable results.}

\indent{ A feature-based method is a popular approach in image classification problems, there are many advantages and results with a good performance in many related pieces of research, as in \cite{feature2018select}, \cite{hal2008features}. The main idea of this method is to process the only relevant information of an image, and to reduce the feature space dimension as much as possible. Likewise, dimensionality reduction is the way of solving the curse of dimensionality problem formulated by Richard Bellman in 1957 \cite{Bellman1957dimcurse}. The suggested method is described in \cite{features2003extraction}.}

\indent{ The goal of our work is to implement segmentation and classification of a specific type of datasets for lower processing time and computational complexity. In this paper, we study binary classifier  capabilities on the dataset \cite{giselsson2017public} consisting of images of 12 species and containing the most common weed species in Danish agriculture.}
